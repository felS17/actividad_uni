\documentclass{article}

\title{Actividad Evaluada 5: Git y Github}
\author{Antonio Navarro y Felipe Schneeberger}
\date{}

\begin{document}

\maketitle

\section{Programa: Conversor de código Python-C++}
\subsection{Función}
Este programa, tiene por propósito ser un conversor de código entre los lenguajes de programación Python 3 y C++. El método de uso es simple: Al ingresar un archivo .py a la carpeta contenedora, y renombrarlo como python.py el código eficientemente reescribirá el código para encajar con el syntax propio de C++.\par
El código fue creado para la actividad evaluada 5 del curso IIC1001 de la UC. Pero de todas maneras no se descarta continuar con su desarrollo a futuro.
\subsection{Aplicación práctica}
Las aplicaciones prácticas del código son incontables. Puede ser útil para mejorar la eficiencia de un programa, gracias a que C++ es un lenguaje más rápido en ejecución comparado a Python (debido a un sinfín de razones que no listaré acá, puede Googlearlo si le interesa).\par
Esto permitiría, en teoría, pasar por compilador un código escrito en Python, a diferencia de su naturaleza interpretada.

\subsection{Limitaciones}
Como esta es una actividad que no me evalúa el contenido del código, su desarrollo no está ni siquiera a medias, pero la idea es la que cuenta.\par
Es más fácil mencionar en que casos el código si funciona, más que al contrario, asi que eso haré.\par
\subsubsection{Casos donde SI funciona:}
\begin{itemize}
    \item Traduce efectivamente la inicialización de C++, incluyendo por defecto la librería \textit{iostream} y metiendo todo al \textit{main()}. No hay bugs conocidos al respecto.
    \item Cambia de buena manera el \textit{print} de python, al syntaxis de \textit{printf}, función propia de C++. (Funciona solo con strings)
    \item Actualmente se está trabajando en la función \textit{while}, está dificil conseguirlo, pero el código va por buen camino.
\end{itemize}



\end{document}